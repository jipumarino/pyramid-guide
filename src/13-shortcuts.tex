\chapter{Shortcuts}

\section{In any mode}

Hold \btn{TRACK} to activate the following shortcuts:

\begin{itemize}
\item \btncolor{pyyellow}{solo}
\item \btncolor{pyyellow}{midi channel}
\item \btncolor{pyyellow}{zoom}
\item \btncolor{pyyellow}{length}
\item \btncolor{pyyellow}{time signature}
\end{itemize}

To change the BPM decimal, turn \encodericon{} while pressing it.

Hold \blackbtn{2ND} + \btn{BPM} and turn the data knob \encodericon{} to change the tempo, which will apply only when you release \blackbtn{2ND}.

Once an encoder is assigned, press it to display its set value.

In \btn{FX manager}, \btn{Assign}, \btn{DISP} or \btn{Settings}, press \btn{BPM} to exit and return to the current mode.

Hold \blackbtn{2ND} and press \btn{rec} to activate the metronome.


\section{Live mode}

Hold \btn{LIVE} and rotate \encodericon{} to change the type of the 8 smartpads.

Hold \btn{STEP} + \btncolor{pygreen}{velocity} and rotate \encodericon{} to change the keypad and smartpads velocity. You can also press \btn{ASSIGN} to enable the ENCODER STEP EDIT feature, in order to set the \btncolor{pygreen}{velocity} with the touchpad X axis.

When the player is stopped \stopicon{} and you press \btn{rec}, Pyramid is waiting for a press on play \playicon{}. Once you press \playicon{} a countdown is launched. After one bar of countdown, recording starts and continues until the end of the track loop, at which point recording then stops and the loop continues to play as normal.

When smartpads = CHORD, \blackbtn{2ND} + rotating \encodericon{} will jump to the very next major/minor tonality. Add complexity by pressing and rotating .

When smartpads = SCALE, \blackbtn{2ND} + rotating \encodericon{} will select the root note of the scale.

\blackbtn{2ND} + play \playicon{} activate the HOLD/RELATCH mode.


\section{Step mode}

Hold \btn{STEP} and turn \encodericon{} to change the Stepmode.

Hold \blackbtn{2ND} and rotate \encodericon{} to scroll the note list or the CC list faster.

Pre-listen to the selected note by pressing \encodericon{}. Pressing \btn{<} or \btn{>} while holding \encodericon{} allows you to jump octaves for a quicker note navigation. You can also press a key \keypadicon{} while holding \encodericon{} to straight select the note number.

In stepmode \btn{NOTE}, you can display the parameters of a step. Hold \btncolor{pygreen}{velocity} and press a filled step to display (and even edit) the step velocity. Same behaviour for \btncolor{pygreen}{length} and \btncolor{pygreen}{width}. You can also display and edit several steps: hold \btncolor{pygreen}{velocity} (or \btncolor{pygreen}{length} or \btncolor{pygreen}{width}) and press the first step together with the last step of the row \stepbystepselectionicon{}.

Double tap a step-by-step pad \stepbystepicon{} to add a low velocity note in order to create accents quickly.

Hold \btncolor{pygreen}{velocity} and slide the touchpad from left to right to quickly set the velocity, the CC value or the FX value.

Hold \btn{<} or \btn{>} to quickly navigate pages.

Press \blackbtn{2ND} and hold \btn{<} or \btn{>} to quickly rotate the track (shift all notes one step left or right).

Press \btn{rec} to switch to MONO EDITION, to have a global view of which of the 16 steps are filled with notes. If you press a filled step, it will remove all notes in this step, whatever the selected note is.

In stepmode \btn{NOTE} and \btn{CHORD}, hold a step \stepbystepicon{} (or a row of steps) and rotate \encodericon{} to transpose it. Hold \btn{rec} and rotate \encodericon{} to transpose the track notes.

In stepmode \btn{CC MESSAGES} and \btn{EFFECTS}, hold a step \stepbystepicon{} (or a row of steps) and rotate \encodericon{} to increase/decrease its value. Hold \btn{rec} and rotate \encodericon{} to transpose the whole automation.

In stepmode \btn{CC MESSAGES} and \btn{EFFECTS}, hold \btn{rec} and slide the touchpad to draw an automation.

Press \btn{ASSIGN} to enable the ENCODER STEP EDIT feature, in order to assign direclty the 5 encoders and the touchpad to shortcuts.

In stepmode \btn{NOTE} and \btn{CC MESSAGES}, double tap \btn{STEP} to activate the filter, in order to scroll only through programmed notes or CC messages. Double tap \btn{STEP} again to quit the filter.


\section{Track mode}

Hold \blackbtn{2ND} and select a track \stepbystepicon{} to mute/unmute it instantly.

Hold \btncolor{pyyellow}{midi channel} and press + rotate \encodericon{} to quickly select the MIDI output of the track.

Hold \btncolor{pyyellow}{length} and press + rotate \encodericon{} to set the bar level with step precision.

Hold \btncolor{pyyellow}{time signature} and rotate \encodericon{} to change the time signature's upper number on the current track.

Pess \btn{rec} in order to don't save the sequence mute states (rec led will be OFF). If you mute or unmute a track, the sequence will not save the mute states.

In \btn{DISP} view, rotate \encodericon{} to scroll the track banks (TR01A > TR07A, TR08A > TR16A, TR01B > TR07B, \ldots)


\section{FX manager mode}

With the effect manager activated, you can still play or add notes in \btn{LIVE} and \btn{STEP} mode, mute/unmute tracks in \btn{TRACK} mode, and even launch sequences in \btn{SEQ} mode.

Rotating \encodericon{} while pressing \blackbtn{2ND} allows you change the position of the selected effect.

In \btn{LIVE} mode, when you are editing an effect with the 5 encoders \encodersicon{}, press \btn{rec} to record directly the effect parameters automations.


\section{Assign mode}

Hold \blackbtn{2ND} and rotate \encodericon{} to scroll the CC list faster.


\section{Save/load mode}

Hold \blackbtn{2ND} and rotate \encodericon{} to move the cursor vertically in the SAVE AS\ldots edition display.

% Vamos en 50
