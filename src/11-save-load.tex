\chapter{Save/Load}

\section{Save/load principles}

To save or load a project, make sure a SD card is inserted, hold \blackbtn{2ND} and press \blackbtn{save/load} to enter the menu:

\fitgraphics{saveload.png}

A project is a whole set of 64 tracks (notes + effects racks + CC messages automations + effect parameters automations), 16 sequences, sequences chain, assignations and settings. Saving or loading a new project will stop the project playback.

\note{Please not remove the SD card while save/load menu is activated. To exit the save/load menu, press \blackbtn{2ND} + \blackbtn{save/load}.}

\note{To exit the Save/load menu, press \blackbtn{save/load} or \blackbtn{2ND}.}


\section{Save}

Select \textbf{SAVE} with \encodericon{} to save the project on the SD card with the current project name. If no project name was defined (NEW PROJECT), \textbf{SAVE} will redirect you to \textbf{SAVE AS}.

\note{Saving an existing project creates a backup file in the SD card, for example ``BACKUP\_PROJECT1'', to always keep an old version of your project. This backup is not displayed on the LOAD project list. If you want to load it, rename the file, for example ``PYRA\_PROJECT1''.}


\section{Save As}

Select \textbf{SAVE AS} with \encodericon{} to save the project with a new name. Then edit the project name with this edition display:

\fitgraphics{save.png}

Select ``<'' to delete the last character.
Select ``SAVE'' to validate.

\tip {Hold \blackbtn{2ND} and rotate \encodericon{}  to move the cursor vertically in the edition display.}


\section{Load}

Select \textbf{LOAD} with \encodericon{} to load a new project on the SD card. This action will delete the current project: you have to save your project before loading an other one.

Select the project to load in the dropdown list with \encodericon{}:

\fitgraphics{load.png}

Press \encodericon{} to validate.


\section{New}

Select \textbf{NEW} to delete everything in the project and restart from scratch.


\section{Project files}

Projects directories are always saved on the SD card with the prefix ``PYRA\_'', on the SD root directory.

You can read the project files on your computer browser:

\fitgraphics{browser.png}

The \textbf{core.pyr} file contains the project settings, the sequences data, assignations and effects racks.

The other files can be deleted:

The track 01A contains MIDI notes (\textbf{track01.mid}), MIDI CC automations (\textbf{CC\_track01.mid}) and effects parameters automations (\textbf{FX\_track01.pyr}).

The track 09A contains only MIDI CC automations (\textbf{CC\_track09.mid}).

The track 10A contains only effects parameters automations (\textbf{FX\_track10.pyr}).

The track 01B contains only MIDI notes (\textbf{track17.mid}).

The track 16D contains only MIDI notes (\textbf{track64.mid}).


\section{Import and export files}

It's easy to import and export MIDI files (notes and CC automations) because Pyramid store these files as standard *.mid data (type 0 MIDI file: single track). All popular DAWs can play and edit MIDI files.

You can import or export from short patterns up to long linear tracks. For example, you can drag-and-drop a *.mid file from the SD card project directory to your favourite software sequencer (e.g. Ableton) and play \& edit this pattern on the DAW:

\fitgraphics{ABLETON.png}

You can also create a MIDI track from scratch on your DAW and export it to an existing SD card project directory, with a valid name. If you want to export the *.mid file (filled with notes) to the Pyramid TR08A, name the file \textbf{track08.mid} (track 01 + bank A = 0*16). For TR01B, name the file \textbf{track17.mid} (track 01 + bank B = 1*16). For TR16D, name the file \textbf{track64.mid} (track 16 + bank D = 3*16).

Pyramid will play this *.mid note pattern in a loop, depending on the track length.

Same operations for a CC messages automation track. If you want to export the *.mid file (filled with CC automations, pitch bend and/or channel pressure) to the Pyramid TR08A, name the file \textbf{CC\_track08.mid} and export it in an existing directory.
