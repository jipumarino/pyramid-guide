\chapter{Step Mode}

The Step Mode is the second way to create rhythms and melodies on the current track. Unlike the Live Mode, in which you perform with the keypad and the smartpads, the Step Mode allows you to ``program'' the track, thanks to step-by-step pads. It's a great way to get the best of your synthesizers and drum machines.

Pyramid includes a 16-step interface, but you are free to extend the length of the track up to 384 bars and the zoom up to 1600\%: the number of steps is almost infinite! The ability to add multiple notes on a single step, modify the velocity, the note width and the offset for each step makes Pyramid one of the most advanced sequencer on the market.

The Step Mode is also a great tool to edit your live recording with the piano roll view.


\section{Screen}

\fitgraphics{STEP-1.png}


\section{Pads}

\fitgraphics{STEP_PADS.png}


\section{Change the Stepmode}

The Stepmode defines the method according to which the steps are filled. Hold \btn{STEP} and turn the data knob \encodericon{} to change the Stepmode, displayed on the screen:

\begin{itemize}
\item \textbf{NOTE} (default)
\item \textbf{CHORD}
\item \textbf{EUCLID}
\item \textbf{CC MESSAGES}
\item \textbf{EFFECTS}
\end{itemize}

\fitthreegraphics{step_note_full.png}{step_chord_full.png}{step_euclid_full.png}
\fittwographics{step_cc_full.png}{step_fx_full.png}

All stepmodes (except \textbf{EUCLID}) can be used together to create a track. For example, you can add chords with the \textbf{CHORD} stepmode, then edit these chords with the \textbf{NOTE} stepmode, create step-lock automations with \textbf{CC MESSAGES} and \textbf{EFFECTS} stepmodes, go back to NOTE stepmode to add extra notes\ldots

The \textbf{EUCLID} stepmode can't be used together with \textbf{NOTE} or \textbf{CHORD} stepmodes because the euclidean algorithm takes exclusive control of the track player.


\section{Note Stepmode}

This mode allows you to program notes by filling steps.

\subsection{Add notes}
Select a note on the screen with \encodericon{} and press a step-by-step pad \padsicon{} to fill it with a note.

\fitgraphics{STEP-NOTE.png}

Press it again to remove the note. You can add multiple notes on a single step. A step filled with the current selected note will softly backlit the corresponding pad.

\tip{Hold  \blackbtn{2ND} and rotate \encodericon{} to scroll the current note list faster.}

\tip{Pre-listen to the selected note by pressing \encodericon{}. Pressing \btn{<} or \btn{>} while holding \encodericon{} allows you to jump octaves for a quicker note navigation. You can also press a key \keypadicon{} while holding \encodericon{} to straight select the note number.}

When you add a note on a step, the step is filled with its \textbf{3 current parameters}: \btncolor{pygreen}{velocity}, \btncolor{pygreen}{note width} and \btncolor{pygreen}{offset}. Default velocity is \textbf{100}, default note width is \textbf{1 step} and default offset is \textbf{0\%}. This means that an added note will fill a step from its start to its end, so the note duration depends on the zoom and the time signature. If \btncolor{pyyellow}{zoom} = 100\% and \btncolor{pygreen}{note width} = 1, the added note will be triggered during a quarter note (because the 16 steps represent 16 quarter notes). If \btncolor{pyyellow}{zoom} = 200\% and \btncolor{pygreen}{note width} = 1, the added note will be triggered during a eighth note (because the 16 steps now represent 16 eighth notes). You can play with the \btncolor{pyyellow}{zoom} shortcut to edit your track with a lot of resolution!

\subsection{Note velocity}

To change the note velocity, hold \btncolor{pygreen}{velocity} and rotate \encodericon{} (from \textbf{0} to \textbf{127}):

\fitgraphics{velocity.png}

If you want to edit the velocity for a row of steps, hold \btncolor{pygreen}{velocity}, press the first step together with the last step of the row \stepbystepicon{} and rotate \encodericon{}.

\tip{Double tap a step-by-step pad \padsicon{} to add a low velocity note in order to create accents quickly.}

\subsection{Note width}

The note will be played during the whole width of a step. Hold \btncolor{pygreen}{note width} and rotate \encodericon{} to change it (from \textbf{1/32} of a step to \textbf{16} steps):

\fitgraphics{width132.png}
\fitgraphics{width16.png}

You can also edit the width of several steps: hold \btncolor{pygreen}{note width}, press the first step together with the last step of the row \stepbystepicon{} and rotate \encodericon{}.

\tip{Using a \btncolor{pygreen}{note width} longer than 1 step is an easy way to create slides on monophonic instruments.}

\subsection{Note offset}

The filled note starts at the beginning of a step. Hold \btncolor{pygreen}{offset} and rotate \encodericon{} to change the position of a note when you fill the step (from \textbf{0\%} to \textbf{99\%} of a step):

\fitgraphics{offset68.png}

You can edit the offset of several steps: hold \btncolor{pygreen}{offset}, press the first together with the last step of the row \stepbystepicon{} and rotate \encodericon{}.

\subsection{Piano roll}
Press \btn{DISP} to view the piano roll sequencer:

\fitgraphics{STEP_DISPLAY.jpg}

\note{An example of a track, programmed with notes of different lengths. You are viewing the page 1 of 4 pages. The current note is C4, the current velocity is 127, the current note length is 4 steps and the current offset is 0\%. If you press a step to add a note, the note will be added with these parameters.}

\subsection{Poly edition / mono edition}

\fittwographics{POLY-EDITION.png}{MONO-EDITION.png}

The \textbf{POLY EDITION} is enabled by default. It's useful to edit tracks controlling polyphonic instruments, in order to add multiple notes on a single step. In \textbf{POLY EDITION}, steps softly backlit only if they are filled with the selected note (the note displayed on the screen). If you  copy  some steps, the \blackbtn{copy} will be effective only for the selected note. Same behavior for the \blackbtn{paste}, the selected step \btncolor{pygreen}{velocity} and the selected step \btncolor{pygreen}{offset}.

To switch to \textbf{MONO EDITION}, press the \btn{rec} pad (rec led will flash). It's useful to edit tracks controlling monophonic instruments, like a classic step-by-step sequencer. Whatever the selected note is, steps softly backlit if they are filled. You can't add multiple notes on a single step: all notes of the track are stacked. If you \blackbtn{copy}  some steps, the copy will be effective for all the notes in selection. Same behavior for the \blackbtn{paste}, the selected step \btncolor{pygreen}{velocity} and the selected step \btncolor{pygreen}{offset}.

\tip{Switch to \textbf{MONO EDITION} to have a global view of which of the 16 steps are filled with notes. If you press a filled step, it will remove all notes in this step, whatever the selected note is.}

\subsection{Track transpose}

In \textbf{POLY EDITION}, hold \btn{rec} and rotate \encodericon{} to transpose only the current notes of the track (e.g. all C4 notes).

\fitgraphics{transpose_poly.png}

In \textbf{MONO EDITION}, hold \btn{rec} and rotate \encodericon{} to transpose all notes of the track.

\fitgraphics{transpose_mono.png}

\subsection{Step transpose (analog step-by-step style)}

In \textbf{POLY EDITION}, hold a step \padsicon{} and rotate \encodericon{} to transpose only the current note of the step.

\fitgraphics{step_poly.png}

In \textbf{MONO EDITION}, hold a step \padsicon{} and rotate \encodericon{} to transpose all notes of the step.

\fitgraphics{step_mono.png}

\tip{You can also transpose a row of steps \stepbystepicon{}, by pressing the first and last step, and rotating \encodericon{}.}


\section{Chord Stepmode}

This mode is similar to the Note Stepmode except the steps are filled with chords.

\subsection{Add chords}

Select a chord degree on the screen with \encodericon{}:

\fitgraphics{STEP-CHORD.png}

\tip{Hold \blackbtn{2ND} and rotate \encodericon{} to change the chord tonality.}

Press a step-by-step pad \padsicon{} to fill it with a chord. A step filled with a chord is semi-highlighted.

\tip{Pre-listen to the selected chord by pressing \encodericon{}.}

You can add only one chord per step. If you work on a step that has already been filled, the previous chord will be replaced with the new one.

There are 7 chord degrees across the 10 octaves:\\
\textbf{I  II  III  IV  V  VI  VII}\\
These are the same chords that you can find on the Live Mode's chord smartpads (chord degree I is smartpad 1, chord degree III is smartpad 3, etc.)

The chord tonality (by default \textbf{CMAJ}) and the chord complexity (by default \textbf{3}) can therefore be set in the Live Mode, with the chord smartpads selected.

\note{Note: Read the Live Mode section for further details about chord generation.}

\subsection{Velocity / note width / offset}
As in Note Stepmode, you can edit the \btncolor{pygreen}{velocity}, \btncolor{pygreen}{note width} and \btncolor{pygreen}{offset} of the chord.

\subsection{Piano roll}

Press \btn{DISP} to view the piano roll sequencer:

\fitgraphics{DISPLAY-CHORD.png}

\note{An example of a track, programmed with chords of different lengths. You are viewing the page 1 of 1 page. The selected chord root note is E4, the current velocity is 127, the current note length is 2 steps and the current offset is 0\%.}

\subsection{Track transpose}

Hold \btn{rec} and rotate \encodericon{} to transpose all notes of the track.

\subsection{Step transpose (analog step-by-step style)}

Hold a step \padsicon{} and rotate \encodericon{} to transpose the chord of the step.

\tip{You can also transpose a row of steps \stepbystepicon{}, by pressing the first and last step, and rotating \encodericon{}.}

\section{Euclid Stepmode}

The Euclidean sequencer relies on one of the most intuitive ways to create uncommon and rich rhythmic patterns.

Originally derived from nuclear physics, and then applied to music theory, the Euclidean algorithm evenly generates distributed notes (fills) among a defined number of available slots (steps).

\fitgraphics{img_pyramid_euclidean_sequencer.png}

This leads to a great number of well-known rhythm patterns, as well as many odd-sounding ones. Almost all traditional rhythms from across the world can be generated with this tool, which is why we decided to implement it into Pyramid.

\note{The EUCLID stepmode takes exclusive control of the track player. You can't use it together with NOTE and CHORD stepmodes. When you change the stepmode, EUCLID is activated at the STEP pad release, and the player of NOTE + CHORD stepmodes is disable. As soon as you switch to NOTE, CHORD, CC MESSAGES or EFFECTS stepmodes, EUCLID is disable again.}

Hold \btncolor{pygreen}{euclid step} or \btncolor{pygreen}{euclid fills} and rotate \encodericon{} to build your pattern. Press \blackbtn{2ND} + \btn{<} or \blackbtn{2ND} + \btn{>} to rotate the pattern.

\fitgraphics{STEP-EUCLID.png}

\note{By default, euclid steps=16 (1 bar) and euclid fills=4.}

Turn \encodericon{} to change the triggered note (for example \textbf{C3\#}) and use \btncolor{pygreen}{note width} to change the gate.

Press \btn{DISP} to view the Euclidean circle:

\fitgraphics{EUCLID.png}

\tip{The filled steps are also displayed on the pads. You can press a pad \padsicon{} to add or remove steps and thus create your own pattern.}

You can create multiple euclidean tracks to create complex rhythmic patterns. Each of them will be 1 bar length or lower, but because the measure size differs, they shift against each other: the loop is ever-changing.

\tip{Decrease the time signature lower number to multiply the euclidean pattern length. By default the time signature of a euclidean track is 4/16 (= 1 bar length) but you can set a 4/8 time signature (2 bars length), a 4/4 time signature (4 bars length)\ldots}


\section{CC messages Stepmode}

This stepmode allows you to create CC automations.

\subsection{Create step-by-step automation}

Select the CC message (0 to 119) you want to automate, or a PITCH BEND, PRESSURE or PROGRAM CHANGE message with \encodericon{} (or hold \blackbtn{2ND} and rotate \encodericon{} to scroll the CC list faster):

\fitgraphics{step_cc_full_value_63.png}

\note{PITCH BEND message is selected, the current value is 63.}

Now you can fill steps with CC messages. To choose the CC value (from 0 to 127), hold \btncolor{pygreen}{velocity} and rotate \encodericon{} (or hold \btncolor{pygreen}{velocity} and slide the touchpad from left to right):

\fitgraphics{step_cc_full_value_127.png}

\note{PITCH BEND message is selected, the current value is 127 (if you press a step, a pitch=127 message will be added).}

Fill some steps with different CC values.

Press \btn{DISP} to display the automation:

\fitgraphics{display-cc-messages.png}

\note{A 16-step automation of the PITCH BEND message. The dotted line represent the current CC value.}

\tip{Press \btn{ASSIGN} to activate ENCODER STEP EDIT. Then you can select the current CC value directly with the touchpad X axis.}

This stepmode can also be used to display and edit automations recorded in live mode, for example a CC10 performed with the touchpad:

\fitgraphics{display-CC-pan.png}

\note{A live recorded automation of the CC10 (PAN) message.}

As in other stepmodes, you can \btncolor{pyyellow}{zoom} to increase the step resolution and navigate the automation pages with \btn{<} and \btn{>}:

\fitgraphics{display-CC-pan-zoom.png}

\note{A live recorded automation of the CC10 (PAN), zoomed.}

You can create an infinite number of automations on the same track, rotate \encodericon{} to select another CC message and program another automation!

\subsection{Draw an automation}

Hold \btn{rec} while sliding the touchpad to draw a CC automation. Very useful to experiment complex automations in a flash!

\fitgraphics{draw_automation.png}

Each new automation drawn will delete the last automation.

\subsection{“Transpose” automation}

Hold \btn{rec} and rotate \encodericon{} to increase or decrease the whole existing CC automation.

\subsection{Create step-lock style automation}

Hold a step \padsicon{} and rotate \encodericon{} to increase or decrease the step automation. Very useful to creates step-lock automations in a flash.

\tip{You can also increase or decrease a row of steps \stepbystepicon{}, by pressing the first and last step, and rotating \encodericon{}.}


\section{Effects Stepmode}

This stepmode allows you to create FX parameters automations.

\subsection{Create step-by-step automation}

First, you need to add at least one effect with the FX manager on the current track, for example an arpeggiator.

In \textbf{EFFECTS} stepmode, select the effect parameter you want to automate with \encodericon{}:

\fitgraphics{step_fx_parameter.png}

\note{ARPEGGIATOR effect: RATE parameter is selected, the current parameter value is 1/4.}

Now you can fill steps with FX parameters. To choose the parameter value, hold \btncolor{pygreen}{velocity} and rotate \encodericon{} (or hold \btncolor{pygreen}{velocity} and slide the touchpad from left to right):

\fitgraphics{step_fx_parameter_32.png}

\note{ARPEGGIATOR effect: RATE parameter is selected, the current parameter value is 1/32 (if you press a step, a rate=1/32 message will be added).}

Fill some steps with different parameters values. Press \btn{DISP} to display the automation:

\fitgraphics{display-FX-arp.png}

\note{A 16-step automation of the ARPEGGIATOR effect: RATE parameter.}

\tip{Press \btn{ASSIGN} to activate ENCODER STEP EDIT. Then you can select the current FX value directly with the touchpad X axis.}

As in other stepmodes, you can \btncolor{pyyellow}{zoom} to increase the step resolution and navigate the automation pages with \btn{<} and \btn{>} to create a complex automation:

\fitgraphics{display-FX-quant.png}

\note{A 64-step automation of the QUANTIZER effect: GRID parameter}

This stepmode can also be used to display and edit automations recorded in live mode, for example a swing automation performed with an encoder:

\fitgraphics{display-FX-swing.png}

You can create an infinite number of automations on the same track, rotate \encodericon{} to select another effect parameter and program another automation!

\subsection{Draw an automation}

Hold \btn{rec} while sliding the touchpad to draw a FX automation. Very useful to experiment complex automations in a flash!

Each new automation drawn will delete the last automation.

\subsection{“Transpose” automation}

Hold \btn{rec} and rotate \encodericon{} to increase or decrease the whole existing FX automation.

\subsection{Create step-lock style automation}

Hold a step \padsicon{} and rotate \encodericon{} to increase or decrease the step automation. Very useful to creates step-lock automations in a flash.

\tip{You can also increase or decrease a row of steps \stepbystepicon{}, by pressing the first and last step, and rotating \encodericon{}.}


\section{Track player: pages, zoom and length}

The track player zone displays data on of the current track:

\begin{itemize}
\item \tracklenicon{} Track length (number of bars \baricon{})
\item \pageposicon{} Page position (viewed bars on the 16 pads)
\item \playerposicon{} Player position (sequencer track progression)
\end{itemize}

Any default track is 4-bar long. Therefore, the track player zone is displaying 4 white bars: they are the 4 bars you can see on the 16 pads (currently viewed page) in Step Mode. These 16 pads represent 16 steps = 16 quarternotes.

\fitgraphics{player_zone_1.png}

If you want to program a longer or shorter track, hold \btn{TRACK} + \btncolor{pyyellow}{length} and rotate \encodericon{} to change the bar number of the track, for example to 16 bars:

\fitgraphics{player_zone_2.png}

Press \btn{<} or \btn{>} to navigate through the pages:

\fitgraphics{player_zone_3.png}

\tip{Hold \btn{<} or \btn{>} to quickly navigate.}

To change the pad resolution, hold \btn{TRACK} + \btncolor{pyyellow}{zoom} and rotate \encodericon{}. For example, 2 bars are displayed in 200\% zoom on a row of 16 pads, also called a page. In 400\% zoom, only 1 bar is displayed.

\fitgraphics{player_zone_4.png}

\tip{Microscope mode: 1600\% zoom provides you with the highest degree of precision to edit your track step by step.}

\note{Note: all these examples about bars may be applied to a standard 4/4 time signature, because a 4/4 bar is always scaled on a 16-step page. For any other time signature, please consider \baricon{} and \pageposicon{} respectively as pages and viewed pages.}


\section{Rotate}

In stepmodes \textbf{NOTE}, \textbf{CHORD} and \textbf{EUCLID}, press \blackbtn{2ND} + \btn{<} or \blackbtn{2ND} + \btn{>}  to shift the track one step left or right.

As zooming changes the track resolution (step length), you will be able to rotate your track with a growingly good precision as you zoom in.

\tip{Press \blackbtn{2ND} and hold \btn{<} or \btn{>} to quickly rotate.}


\section{Copy/Paste}

Press \blackbtn{2ND} + \blackbtn{copy} + any step \padsicon{} to copy all the notes or automations contained in a step.

You can copy multiple steps by pressing  \blackbtn{2ND} + \blackbtn{copy} + the first and the last steps \stepbystepicon{} of your selection.

Then you can paste notes and automations in the current track or in another track. Press  \blackbtn{2ND} + \blackbtn{paste} + the step \padsicon{}.

\note{Note: copy/paste is not available in \textbf{EUCLID} stepmode.}


\section{Delete}

Press \blackbtn{2ND} + \blackbtn{delete} + any step \padsicon{} to delete all the notes contained in the step.

\note{Note: delete is not available in \textbf{EUCLID} stepmode.}


\section{Encoder step edit}

At startup, the 5 encoders \encodersicon{} can be used to assign a CC or FX parameter, in every mode. But they can do more. In STEP mode, you can decide to assign directly the 5 encoders and the touchpad to step shortcuts.

Simply press \btn{ASSIGN} to enable the ``encoder step edit'' feature:

\fitgraphics{encoder_step_edit.png}

It's a great way to edit your notes and automations in a flash, you no longer have to rotate \encodericon{} while pressing a green shortcut.

In stepmode \textbf{NOTE}, encoders \& touchpad will control:

\begin{itemize}
\item \blackbtn{1}\btncolor{pygreen}{note}
\item \blackbtn{2}\btncolor{pygreen}{note octave}
\item \blackbtn{3}\btncolor{pygreen}{note velocity}
\item \blackbtn{4}\btncolor{pygreen}{note width}
\item \blackbtn{5}\btncolor{pygreen}{note offset}
\item \blackbtn{touchpad X}\btncolor{pygreen}{note velocity}
\end{itemize}

In stepmode \textbf{CHORD}, encoders \& touchpad will control:

\begin{itemize}
\item \blackbtn{1}\btncolor{pygreen}{chord degree}
\item \blackbtn{2}\btncolor{pygreen}{chord tonality}
\item \blackbtn{3}\btncolor{pygreen}{chord velocity}
\item \blackbtn{4}\btncolor{pygreen}{chord width}
\item \blackbtn{5}\btncolor{pygreen}{chord offset }
\item \blackbtn{touchpad X}\btncolor{pygreen}{chord velocity}
\end{itemize}

In stepmode \textbf{EUCLID}, encoders \& touchpad will control:

\begin{itemize}
\item \blackbtn{1}\btncolor{pygreen}{euclid steps}
\item \blackbtn{2}\btncolor{pygreen}{euclid fills}
\item \blackbtn{3}\btncolor{pygreen}{euclid velocity}
\item \blackbtn{4}\btncolor{pygreen}{euclid gate length}
\item \blackbtn{5}\btncolor{pygreen}{rotate }
\item \blackbtn{touchpad X}\btncolor{pygreen}{euclid velocity}
\end{itemize}

In stepmode \textbf{CC}, encoders \& touchpad will control:

\begin{itemize}
\item \blackbtn{3}\btncolor{pygreen}{CC value}
\item \blackbtn{touchpad X}\btncolor{pygreen}{CC value}
\end{itemize}

In stepmode \textbf{EFFECTS}, encoders \& touchpad will control:

\begin{itemize}
\item \blackbtn{3}\btncolor{pygreen}{FX value}
\item \blackbtn{touchpad X}\btncolor{pygreen}{FX value}
\end{itemize}

\tip{In \btn{LIVE} mode, you can also use the touchpad to set the velocity of the keyboard and the smartpads.}

Press \btn{ASSIGN} again to disable these shortcuts.


\section{Filter edition}

In stepmode NOTE and stepmode CC MESSAGES, double tap \btn{STEP} to activate the filter, in order to scroll only through programmed notes or CC messages. A popup is displayed for a few time:

\fitgraphics{FILTER_ON.png}

For example, in NOTE stepmode, rotating \encodericon{} will jump to the next programmed note (e.g. D4), useful if use a wide range of notes. In CC MESSAGES stepmode, rotating \encodericon{} will jump to the next CC automation (e.g. Program Change).

When the filter is activated, a circle pictogram is displayed next to the stepmode name:

\fitgraphics{FILTER.png}

Double tap \btn{STEP} to exit the filter mode.
